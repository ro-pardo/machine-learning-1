% teilt dem Editor mit, dass er latin1 als encoding verwenden soll
% !TeX encoding = latin1

% festlegen und modifizieren der Dokumentenklasse
\documentclass[
	pdftex,						% PDFtex verwenden
	a4paper, 						% A4 Format benutzen
	12pt, 							% Schriftgr??e 12 f?r normalen Text
	headsepline						% Linie nach Kopfzeile
]{scrartcl}							% KOMA-Script Dokumentklasse "scrartcl" verwenden

% einbinden der Pakete

\usepackage{hyperref}
\usepackage{wrapfig}					% Text neben Bild
\usepackage{amsmath} 					% Paket f?r Formelumgebungen
\usepackage{amssymb} 					% Paket f?rMathe-Symbole

\usepackage{physics}                    %für Bra und Ket Notation

\usepackage{array}       					% Paket zum Erweitern der Tabelleneigenschaften
\usepackage{graphicx}  					% Paket um Grafiken einbetten zu k?nnen
\usepackage{color}       					% Paket f?r Farben im PDF (Seitenfarbe und Textfarbe)
\usepackage{makeidx}   					% Paket f?r die Indexerstellung.
\usepackage[T1]{fontenc} 
\usepackage[utf8]{inputenc}				% Paket f?r Schriftart, Verwenden von T1 Fonts in Ausgabe zur Darstellung von Umlauten
\usepackage{url}     					% Paket zur richtigen Darstellung von URLs
\usepackage[a4paper,left=2.5cm,right=2.0cm]{geometry} % Paket zum Einstellen der Seitenr?nder
\usepackage[ngerman]{babel} 				% Paket f?r Zeichensetzung und Worttrennung, deutsche Zeichensetzung verwenden
  					% Paket f?r Vektorpfeile: \vv{x}

%\usepackage{scrpage2} 
\usepackage{subfigure} 					%Bilder nebeneinander
\usepackage{caption} 					% Paket um Beschriftung f?r Gleitobjekte anzupassen
	\setlength{\captionmargin}{10pt} 		% extra Abstand rechts und links f?r Beschriftung
	\renewcommand{\captionfont}{\footnotesize}	% Beschriftung ist in Fu?notengr??e
			% Paket f?r erweiterte Positionierung von Gleitobjekten, Gleitobjekte d?rfen Abschnitt nicht verlassen
\usepackage{float}						% Paket zur erweiterung der Gleitobjektfunktionalit?ten, erm?glicht feste Positionierung mit \begin{table}[H]
\usepackage{epstopdf}			%Paket zur konvertierung von tif dateien zu png
\epstopdfDeclareGraphicsRule{.tif}{png}{.png}{convert #1 \OutputFile}
\AppendGraphicsExtensions{.tif}
% Setzen globaler Parameter
\graphicspath{{bilder/}} 					% legt globalen Pfad f?r Bilddateien fest
\setlength{\parindent}{0mm} 				% unterdr?ckt den Absatzeinzug


% passt Kopfzeile an
\pagestyle{myheadings}


%\usepackage{scrpage2} 
%\clearscrheadfoot
%\pagestyle{scrheadings}
%\automark[section]{section}
%\ihead{\rightmark}
%\ohead{\leftmark}
%\cfoot{\pagemark}


\begin{document}

% f?gt die Titelseite ein
\begin{titlepage}

\huge\textsc{Machine Learning 1}
\vspace{0,5cm}

\begin{minipage}{0.7\textwidth}
\large\textsc{Technical University Berlin\\ Winter semester 2020/21}
\end{minipage}
\begin{minipage}{0.3\textwidth}
\begin{flushright}
		\includegraphics[width=2cm]{pictures/tu-logo.jpg}
	\end{flushright}
\end{minipage}

\begin{center}

\vspace{0,5cm}
\hline

\vspace{2cm}
\Large\textbf{\textsc{}}
 \Large\textbf{\textsc{Exercise Sheet 2}}
\\
\vspace{0,7cm}
\small(16. November 2020)
\end{center}

\begin{table}[H]
\centering
\begin{tabular}{r l l}
								&								&			\\


\textbf{Students:} 	

								&	Mattes Ohse				 	& 337356        \\	
								&	Florian Ebert		 		& 391837 		\\
								&	Rodrigo Alexis Pardo Meza		 					&			    \\
								&   Bertty Contreras Rojas& \\



\end{tabular} 
\end{table}  
\vspace{3cm}
\normalsize
\begin{center}
   
\end{center}


% Inhaltsverzeichnis
\pagenumbering{Roman} 	% r?mische Zahlen (I,II,III,IV,...) f?r Inhaltsverzeichnis
\newpage

%\tableofcontents		% Inhaltsverzeichnis
%\listoffigures 		 	% Abbildungsverzeichnis

% der Rest ist arabisch (1,2,3,4,...) numeriert:
\pagebreak
\pagenumbering{arabic}	% aribische Numerierung
\setcounter{page}{1}	% setzt den Seitenz?hler auf "1"

% zusammenf?gen der Abschnitte
\section{Exercise 1: Estimating the Bayes Error}
\subsection{a}
The Bayes Formular is given by:
\begin{center}
    \begin{equation}
        P(w_{j})=\frac{P(x\mid w_{j})}{p(x)}.
    \label{eq:Bayesformula}
    \end{equation}
\end{center}

and the Bayes Error is given by:
\begin{center}
    \begin{equation}
        P(error)=min[P(w_{1}|x),P(w_{2}|x)].
        \label{eq:error}
    \end{equation}    
\end{center}

We now wanna show, that the full error can be upper-bounded as:

\begin{center}
    \begin{equation}
        P(error) &\leq\int\frac{2}{\frac{1}{P(w_{1}\mid x)}+\frac{1}{P(w_2|x)}} p(x) dx.
        \label{eq:1a}
    \end{equation}
\end{center}

We can assume that $P(w_{1}|x)\geq P(w_{2}|x)$, so that (\ref{eq:error}) becomes to $ P(error)=P(w_{2}|x)$. Now we can use (\ref{eq:1a}), under consideration that we are integrating on both sides for the same variable what means that we can ignore the integralsigns:

\begin{align*}
&\Rightarrow& P(error) &\leq\frac{2}{\frac{1}{P(w_{1}\mid x)}+\frac{1}{P(w_2|x)}} p(x)\\
&\Leftrightarrow& P(w_{2}|x)p(x)&\leq\frac{2}{\frac{1}{P(w_{1}\mid x)}+\frac{1}{P(w_2|x)}} p(x)\\
&\Leftrightarrow& P(w_{2}|x)&\leq\frac{2}{\frac{1}{P(w_{1}\mid x)}+\frac{1}{P(w_2|x)}}\\
&\Leftrightarrow& \frac{P(w_{2}|x)}{P(w_{1}|x)}+\frac{P(w_{2}|x)}{P(w_{2}|x)}&\leq2\\
&\Leftrightarrow& \frac{1}{P(w_{1}|x)}+\frac{1}{P(w_{2}|x)}&\leq\frac{2}{P(w_{2}|x)}\\
&\Leftrightarrow& \frac{1}{P(w_{1}|x)}&\leq\frac{1}{P(w_{2}|x)}\\
\\
&\Leftrightarrow& P(w_{1}|x)&\geq P(w_{2}|x).\\
\end{align*}
And that improves (\ref{eq:1a})






\subsection{b}
We want to show that for the distributions

\begin{center}
    \begin{equation}
        p(x|w_{1})=\frac{\pi^{-1}}{1+(x-\mu)^{2}},\text{ } p(x|w_{2})=\frac{\pi^{-1}}{1+(x+\mu)^{2}}
    \end{equation}
\end{center}
that the Bayes error can be upper-bounded by:

\begin{center}
\begin{equation}
    P(error)\leq\frac{2P(w_{1})P(w_{2})}{\sqrt{1+4\mu^{2}P(w_{1})P(w_{2})}}
\end{equation}
\end{center}

\begin{align*}
\Rightarrow P(error) &\leq\int\frac{2}{\frac{1}{P(w_{1}\mid x)}+\frac{1}{P(w_2|x)}} p(x) dx\\
&=\int\frac{2}{\frac{p(x)}{p(x\mid w_{1})\cdot P(w_{1})}+\frac{p(x)}{p(x|w_{2})\cdot P(w_{2})}} p(x) dx\\
&=\int\frac{2}{\frac{1}{p(x\mid w_{1})\cdot P(w_{1})}+\frac{1}{p(x|w_{2})\cdot P(w_{2})}} dx\\
&=\int\frac{2}{\frac{1}{\frac{\pi^{-1}}{1+(x-\mu)^{2}}P(w_{1})}+\frac{1}{\frac{\pi^{-1}}{1+(x+\mu)^{2}}P(w_{2})}}dx\\
&=\int\frac{2}{\frac{\pi((x-\mu)^{2}+1)}{P(w_{1})}+\frac{\pi((x+\mu)^{2}+1)}{P(w_{2})}}dx\\
&=\int\frac{2}{\frac{\pi(P(w_{2})((x-\mu)^{2}+1)+P(w_{1})((x+\mu)^{2}+1))}{P(w_{1})P(w_{2})}}\\
&=\int\frac{2\cdot P(w_{1})P(w_{2})}{\pi(P(w_{2})((x-\mu)^{2}+1)+P(w_{1})((x+\mu)^{2}+1))}\\
\\
&=\frac{2\cdot P(w_{1})P(w_{2})}{\pi}\int\frac{1}{P(w_{2})(x-\mu)^{2}+P(w_{2})+P(w_{1})(x+\mu)^{2}+P(w_{1})}dx\\
\\
&\text{Use } (x-\mu)^{2}=x^{2}-2x\mu+\mu^{2} \text{ and } (x+\mu)^{2}=x^{2}+2x\mu+\mu^{2}\\
\end{align*}
\\
\begin{align*}
=&\frac{2\cdot P(w_{1})P(w_{2})}{\pi}\\
&\cdot\int\frac{dx}{x^{2}P(w_{2})-2x\mu P(w_{2})+\mu^{2}P(w_{2})+P(w_{2})+x^{2}P(w_{1})+2x\mu P(w_{1})+\mu^{2}P(w_{1})+P(w_{1})}\\
\\
&=\frac{2\cdot P(w_{1})P(w_{2})}{\pi}\\
&\cdot\int\frac{dx}{x^{2}(P(w_{2})+P(w_{1}))+x2\mu(P(w_{1})-P(w_{2}))+\mu^{2}(P(w_{2})+P(w_{1}))+P(w_{2})+P(w_{1})}
\end{align*}

\begin{align*}
&\text{We say: }\\
&a=P(w_{1})+P(w_{2}), \\
&b=2\mu(P(w_{1}-P(w_{2}))\\ 
&c=\mu^{2}(P(w_{1})+P(w_{2}))+P(w_{1})+P(w_{2})\\
&\text{and we use: } \int\frac{1}{ax^2+bx+c}dx=\frac{2\pi}{\sqrt{4ac-b^{2}}}.\\
%&\text{Now we take a look at 4ac:}\\
\end{align*}
\begin{align*}
\text{Now we take a look at 4ac:}\\
4ac=&4[P(w_{1})+P(w_{2})]\cdot[\mu^{2}(P(w_{1})+P(w_{2}))+P(w_{1})+P(w_{2})]\\
=&4(\mu^{^2}+1)[P(w_{1})+P(w_{2})]^{2}.\\
\text{With that and the result of $b^{2}$}\\
b^{2}=&4\mu(P(w_{1})-P(w_{2}))^{2}= 4\mu^{2}[P^{2}(w_{1})-2P(w_{1})P(w_{2})+P^{2}(w_{2})]\\
\text{becomes the integral above to:}
\end{align*}

\begin{align*}
\Rightarrow&\frac{2P(w_{1})P(w_{2})}{\pi}\\
&\cdot\frac{2\pi}{\sqrt{4(\mu^{^2}+1)[P(w_{1})+P(w_{2})]^{2}-4\mu^{2}[P^{2}(w_{1})-2P(w_{1})P(w_{2})+P^{2}(w_{2})]}}\\
\\
=&\frac{2P(w_{1})P(w_{2})}{\sqrt{4\mu^{2}P(w_{1})P(w_{2})+2P(w_{1})P(w_{2})+P(w_{1})+P(w_{2})}}\vert(\text{use: $(a+b)^{2}=a^{2}+2ab+b^{2}$)}\\
\\
=&\frac{2P(w_{1})P(w_{2})}{\sqrt{4\mu^{2}P(w_{1})P(w_{2})+(P(w_{1})+P(w_{2}))^{2}}}\vert(\text{We know that: $P(w_{1})+P(w_{2})=1$})\\
\\
=&\frac{2P(w_{1})P(w_{2})}{\sqrt{1+4\mu^{2}P(w_{1})P(w_{2})}}
\end{align*}
This proves that the full error can be upper-bounded as:

\begin{center}
\begin{equation}
    P(error)\leq\frac{2P(w_{1})P(w_{2})}{\sqrt{1+4\mu^{2}P(w_{1})P(w_{2})}}
\end{equation}
\end{center}




\subsection{c}
For the two cases whether the data is low-dimensional or the data is high-dimensional we can find the upper-bounds by using the methods Chernoff Bound or Bhattacharyya Bound. Both of these methods are numerical. The functions of these integrals are usually not continuous, so that the bounds determined in this way are often tighter than the bounds determined analytically. In high-dimensional cases, Bhattacharyya's method should be better because it is less computationally expensive than the Chernoff Bound method.  
\newpage

\section{Exercise 2: Bayes Decision Boundaries}
\subsection{a}
Univariate Laplacian Distrubution:
$$p(x\mid w_1) = \frac{1}{2\sigma}\exp\left(-\frac{\mid x - \mu \mid}{\sigma}\right) \textrm{ and } p(x\mid w_2) = \frac{1}{2\sigma}\exp\left(-\frac{\mid x + \mu \mid}{\sigma}\right)$$
The optimal decision is to always predict the first class if \forall x \in \mathbb{R}:   $P(w_1\mid x) > P(w_2 \mid x)$.
\begin{align*}
&& P(w_1 \mid x) &> P(w_2 \mid x) \\
&\Leftrightarrow &\frac{p(x \mid w_1)P(w_1)}{p(x)} &> \frac{p(x \mid w_2)P(w_2)}{p(x)} \\
&\Leftrightarrow &p(x\mid w_1)P(w_1) &> p(x \mid w_2)P(w_2) \\
&\Leftrightarrow &\frac{1}{2\sigma}\exp\left(-\frac{\mid x - \mu \mid}{\sigma}\right)P(w_1) &> \frac{1}{2\sigma}\exp\left(-\frac{\mid x + \mu \mid}{\sigma}\right)P(w_2) \\
&\Leftrightarrow &\exp\left(-\frac{\mid x - \mu \mid}{\sigma}\right)P(w_1) &> \exp\left(-\frac{\mid x + \mu \mid}{\sigma}\right)P(w_2) \\
&\Leftrightarrow &\ln(\exp\left(-\frac{\mid x - \mu \mid}{\sigma}\right)P(w_1)) &> \ln(\exp\left(-\frac{\mid x + \mu \mid}{\sigma}\right)P(w_2)) \\
&\Leftrightarrow &-\frac{\mid x - \mu \mid}{\sigma} + \ln(P(w_1)) &> -\frac{\mid x + \mu \mid}{\sigma} + \ln(P(w_2)) \\
&\Leftrightarrow & \ln(P(w_1)) - \ln(P(w_2)) &> \frac{\mid x - \mu \mid}{\sigma} - \frac{\mid x + \mu \mid}{\sigma} \\
&\Leftrightarrow & \ln\left(\frac{P(w_1)}{P(w_2)}\right) &> \frac{\mid x - \mu \mid - \mid x + \mu \mid}{\sigma} \\
&\Leftrightarrow & \sigma\ln\left(\frac{P(w_1)}{P(w_2)}\right) &> \mid x - \mu \mid - \mid x + \mu \mid
\end{align*}
Three cases for the absolute values:
\begin{enumerate}
	\item $x \le \mu$ and $-x \le \mu$ \\
		\begin{align*}
			\sigma\ln\left(\frac{P(w_1)}{P(w_2)}\right) &> -(x - \mu) - (x + \mu ) \\
			&> -2x \\
			- \frac{\sigma}{2}\ln\left(\frac{P(w_1)}{P(w_2)}\right) &< x
		\end{align*}	
		
	\item $x \le \mu$ and $-x > \mu$ \\
		\begin{align*}
			\sigma\ln\left(\frac{P(w_1)}{P(w_2)}\right) &> -(x - \mu) + x + \mu \\
			&> 2\mu
		\end{align*}
	\item $x > \mu$ \\
		\begin{align*}
			\sigma\ln\left(\frac{P(w_1)}{P(w_2)}\right) &> (x - \mu) - (x + \mu ) \\
			&> -2\mu
		\end{align*}
\end{enumerate}
Possible value-tuples where always the first class will be choosen:
\begin{align*}
\{ (P(w_1), P(w_2), \mu, \sigma) \mid &(P(w_1) > P(w_2)) \\
&\land (P(w_1) = 1-P(w_2)) \\
&\land \sigma>0 \land \mu>0 \\ 
&\land ((\sigma\ln\left(\frac{P(w_1)}{P(w_2)}\right)>-2x \land x \le \mu \land -x \le \mu) \\
&\lor (\sigma\ln\left(\frac{P(w_1)}{P(w_2)}\right)> 2\mu \land x \le \mu \land -x > \mu) \\
&\lor(\sigma\ln\left(\frac{P(w_1)}{P(w_2)}\right)> -2\mu) \land x > \mu)\}
\end{align*}

\subsection{b}
Univariate Gaussian Distrubution:
$$p(x\mid w_1) = \frac{1}{\sigma\sqrt{2\pi}}\exp\left(-\frac{(x - \mu)^2}{2\sigma^2}\right) \textrm{ and } p(x\mid w_2) = \frac{1}{\sigma\sqrt{2\pi}}\exp\left(-\frac{(x + \mu)^2}{2\sigma^2}\right)$$
The optimal decision is to always predict the first class if \forall x \in \mathbb{R}:   $P(w_1\mid x) > P(w_2 \mid x)$.
\begin{align*}
&& P(w_1 \mid x) &> P(w_2 \mid x) \\
&\Leftrightarrow &\frac{p(x \mid w_1)P(w_1)}{p(x)} &> \frac{p(x \mid w_2)P(w_2)}{p(x)} \\
&\Leftrightarrow &\frac{1}{\sigma\sqrt{2\pi}}\exp\left(-\frac{(x - \mu)^2}{2\sigma^2}\right)P(w_1) &> \frac{1}{\sigma\sqrt{2\pi}}\exp\left(-\frac{(x + \mu)^2}{2\sigma^2}\right)P(w_2) \\
&\Leftrightarrow &\exp\left(-\frac{(x - \mu)^2}{2\sigma^2}\right)P(w_1) &> \exp\left(-\frac{(x + \mu)^2}{2\sigma^2}\right)P(w_2) \\
&\Leftrightarrow &-\frac{(x - \mu)^2}{2\sigma^2} + \ln\left(P(w_1)\right) &> -\frac{(x + \mu)^2}{2\sigma^2} + \ln\left(P(w_2)\right) \\
&\Leftrightarrow &\ln\left(\frac{P(w_1)}{P(w_2}\right) &> \frac{(x - \mu)^2 - (x + \mu)^2}{2\sigma^2}\\
&\Leftrightarrow &2\sigma^2\ln\left(\frac{P(w_1)}{P(w_2)}\right) &> x^2 - 2x\mu + \mu^2 - x^2 - 2x\mu - \mu^2 \\
&\Leftrightarrow &2\sigma^2\ln\left(\frac{P(w_1)}{P(w_2)}\right) &> -4x\mu \\
&\Leftrightarrow &\frac{2\sigma^2}{-4\mu}\ln\left(\frac{P(w_1)}{P(w_2)}\right) &< x
\end{align*}
Possible value-tuples where always the first class will be choosen:
\begin{align*}
\{ (P(w_1), P(w_2), \mu, \sigma) \mid &(P(w_1) > P(w_2)) \\
&\land (P(w_1) = 1-P(w_2)) \\
&\land \sigma>0 \land \mu>0 \\ 
&\land (\frac{2\sigma^2}{-4\mu}\ln\left(\frac{P(w_1)}{P(w_2)}\right) < x) \}
\end{align*}
\end{document}